\documentclass[12pt]{article}

%DO NOT CHANGE ANYTHING IN THIS PREAMBLE!

\usepackage[a4paper,margin=3cm,innermargin=3cm]{geometry}

\usepackage{needspace}
\usepackage{marginnote}
\usepackage{latexsym,amssymb}

\usepackage{amsmath,amsfonts,bbm, amscd}

\usepackage{color}
\usepackage{mathtools, tensor}


\usepackage{hyperref}


\newenvironment{conf-abstract}[4][]{
  \needspace{10\baselineskip}
  \begin{center}
    { \renewcommand\textsuperscript[1]{}
      \phantomsection\addcontentsline{toc}{section}
      {\texorpdfstring{{\textbf{#3}} \emph{#2}}{#2 (#3)}}
    }
    {{\large\bfseries #2}
    \marginnote{#1}
    \par}
    \medskip
    {#3\par}
    \smallskip
    {\small #4\par}
  \end{center}
}{%
  \bigskip
  \hrule
  \bigskip
}


\pagestyle{plain}
\setlength\parindent{0pt}
\setlength\parskip{3pt}

\begin{document}

% Please do not use ANY self-defined macros!!


\begin{conf-abstract}
{Recent advances in the theory of the Piffle}
{Alfonso Benedetto Smith}
{University of  \dots}


A.C.\ Jones in his paper [1], first defined a Biffle to be a non-definite Boffle and asked if every Biffle was reducible. 
C.D.\ Brown in [2], answered in part this question by defining a Wuffle to be a reducible Biffle and he was then able 
to show that all Wuffles were reducible. 

H.\ Green, P.\ Smith, D.\ Jones, in their review of Brown's paper, suggested the name Woffle for any Wuffle other than the nontrivial 
Wuffle and conjectured that the total number of Woffles would be at least as great as the number so far known to exist. 
They asked if this conjecture was the strongest possible.

T. Brown in [3], dedicated to the honor of R.S.\ Green on his 23rd birthday, defined a Piffle to be an 
infinite multi-variable sub-polynormal Woffle which does not satisfy the lower regular Q-property. 
He states, but was unable to prove, that there was at least a finite number of Piffles.

T.\ Smith, L.\ Jones, R.\ Brown, A.\ Green in their collected works [4], 
showed that all bi-universal Piffles were strictly descending and conjectured that to prove a stronger result would be harder.

It is this conjecture which motivated the present talk. 

% If you report on joint work, add a sentence at the end of the abstract, but do not enter more than one name in the author-brackets above.


%Bibliography, if needed. Please using sparingly.
\vskip1em
{\bf References}
\begin{itemize}
\item[[1\!\!\!]] A.C.\ Jones, A Note on the Theory of Boffles, Proceedings of the National Society, 13 (1984).
\item[[2\!\!\!]] C.D.\ Brown, On a paper by A.C.Jones, Biffle, 24 (1991).
\item[[3\!\!\!]] T.\ Brown, A collection of 250 papers on Woffle Theory (Piffle Press, Boffles, 2001).
\item[[4\!\!\!]] T.\ Smith, L.\ Jones, R.\ Brown, A.\ Green, A short introduction to the classical theory of the Piffle, (The New Piffle Press, New York, 2022).
\end{itemize}


\end{conf-abstract}


\end{document}
